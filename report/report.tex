\documentclass[12pt,a4paper]{article}
\usepackage{polyglossia}
\setmainlanguage{greek}
\usepackage{amsmath,amsfonts,amssymb}
\usepackage{graphicx}
\usepackage{xcolor}
\usepackage{hyperref}
\usepackage{geometry}
\geometry{a4paper, margin=2.5cm}
\usepackage{fontspec}
\setmainfont{FreeSerif}  % Εναλλακτικά: Times New Roman ή άλλη κατάλληλη γραμματοσειρά
\setmonofont{FreeMono}
\usepackage{fancyvrb}


\title{\textbf{Ανάλυση Συστήματος Μεταγλωττιστή για τη Γλώσσα Greek++}}
\author{Τμήμα Πληροφορικής}
\date{\today}


\begin{document}


\maketitle


\begin{abstract}
Το παρόν έγγραφο αποτελεί μια ακαδημαϊκή ανάλυση ενός μεταγλωττιστή για τη γλώσσα greek++, μια εκπαιδευτική γλώσσα προγραμματισμού με ελληνική σύνταξη. Παρουσιάζεται αναλυτικά η αρχιτεκτονική του συστήματος, που αποτελείται από τον λεκτικό αναλυτή και τον συντακτικό αναλυτή. Εξετάζεται η διαδικασία ανάλυσης λεκτικών μονάδων και σύνταξης, καθώς και παραδείγματα κώδικα που δείχνουν τη λειτουργικότητα του μεταγλωττιστή.
\end{abstract}


\tableofcontents
\newpage


\section{Εισαγωγή}
Η γλώσσα greek++ είναι μια εκπαιδευτική γλώσσα προγραμματισμού που υποστηρίζει βασικές δομές όπως συναρτήσεις, διαδικασίες, βρόχους και δηλώσεις υπό συνθήκη. Οι εντολές της γλώσσας είναι στα ελληνικά, προσφέροντας έναν ευκολότερο τρόπο εκμάθησης για ελληνόφωνους χρήστες.


Η ανάλυση εστιάζει στη λειτουργία του μεταγλωττιστή, ο οποίος αποτελείται από:
\begin{itemize}
\item Τον \textbf{λεκτικό αναλυτή} (Lexer), που διαχωρίζει τον πηγαίο κώδικα σε λεκτικές μονάδες (tokens).
\item Τον \textbf{συντακτικό αναλυτή} (Parser), που ελέγχει αν ο κώδικας ακολουθεί τους κανόνες της γλώσσας.
\end{itemize}


\section{Λεκτικός Αναλυτής}
Ο λεκτικός αναλυτής αναγνωρίζει τις βασικές λεκτικές μονάδες της γλώσσας, όπως λέξεις-κλειδιά, αναγνωριστικά, αριθμούς, τελεστές και σύμβολα ομαδοποίησης. Ένα παράδειγμα tokenization δίνεται παρακάτω:


\begin{Verbatim}[frame=single, fontsize=\small]
tokenize("εάν x > 5 τότε γράψε x;")
[('KEYWORD', 'εάν'), ('IDENTIFIER', 'x'), ('RELATIONAL_OPERATOR', '>'),
('NUMBER', '5'), ('KEYWORD', 'τότε'), ('KEYWORD', 'γράψε'),
('IDENTIFIER', 'x'), ('SEPARATOR', ';')]
\end{Verbatim}





Ο αναλυτής διαβάζει τον πηγαίο κώδικα χαρακτήρα προς χαρακτήρα και ταξινομεί τα στοιχεία του. Εντοπίζει επίσης σχόλια μέσα σε αγκύλες {...}, και διαχειρίζεται λευκούς χαρακτήρες.


\section{Συντακτικός Αναλυτής}
Ο συντακτικός αναλυτής χρησιμοποιεί τη γραμματική της greek++ και επαληθεύει ότι ο κώδικας είναι συντακτικά σωστός. Παραδείγματα των κανόνων που ελέγχει είναι:


\begin{itemize}
\item Αν οι δηλώσεις ακολουθούν το πρότυπο: \texttt{δήλωση x,y: ακέραιος}
\item Αν οι εκχωρήσεις τιμών είναι έγκυρες: \texttt{x := 5 * y + 3;}
\item Αν οι εντολές υπό συνθήκη είναι σωστές:
\end{itemize}


\begin{Verbatim}[frame=single, fontsize=\small]
parse("εάν x < 10 τότε γράψε x; εάν_τέλος")
Parsing completed successfully.
\end{Verbatim}





Ο συντακτικός αναλυτής ακολουθεί κανόνες όπως:


\begin{Verbatim}[frame=single, fontsize=\small]
def if_stat(self):
self.eat(token_value='εάν')
self.condition()
self.eat(token_value='τότε')
self.sequence()
self.elsepart()
self.eat(token_value='εάν_τέλος')
\end{Verbatim}


Η συνάρτηση αυτή ελέγχει αν η εντολή \texttt{εάν} έχει σωστή σύνταξη, διαβάζοντας τα επιμέρους στοιχεία της.


\section{Συμπεράσματα και Μελλοντικές Βελτιώσεις}
Ο μεταγλωττιστής της greek++ παρέχει έναν βασικό τρόπο επεξεργασίας κώδικα με ελληνική σύνταξη. Μελλοντικές βελτιώσεις μπορεί να περιλαμβάνουν:
\begin{itemize}
\item Υποστήριξη περισσότερων τύπων δεδομένων (π.χ., αλφαριθμητικά).
\item Σημασιολογική ανάλυση για την ανίχνευση λανθασμένων χρήσεων μεταβλητών.
\item Βελτιστοποίηση της απόδοσης μέσω caching των tokens.
\end{itemize}


\end{document}

